\section{Advancements in Intelligent Transportation Systems: Modeling, Control, and Optimization}


\large \textbf{Organizers} \normalsize \vspace{2mm} \\
\textbf{Yin  Tong} \\ 
\textit{Southwest Jiaotong University} \vspace{{2mm}} \\
\textbf{Silvia  Siri} \\ 
\textit{University of Genova} \vspace{{2mm}} \\
\textbf{Simona  Sacone} \\ 
\textit{University of Genova}

In the rapidly evolving landscape of urbanization, the challenges associated with transportation become increasingly complex. Intelligent Transportation Systems (ITS) have emerged as a crucial solution to address these challenges, leveraging cutting-edge technologies to optimize various facets of urban mobility. The importance of ITS extends beyond mere convenience; it plays a pivotal role in mitigating traffic congestion, enhancing energy efficiency, and ensuring the safety of commuters. In particular, the development of artificial intelligence technology, sensing, and communication technologies has brought unprecedented innovation to the transportation system. Autonomous driving, the Internet of Vehicles, smart logistics, and more have occurred. Central to the evolution of ITS is the development of novel control and optimization techniques that enhance their functionality, effectiveness, and safety. This session serves to gather researchers and practitioners around the world to explore the advances in control and optimization methods of shaping the future of transportation in smart cities and to preview the next steps on the way. 

\section{Advancements in Modeling, Scheduling, and Control for Autonomous Manufacturing}


\large \textbf{Organizers} \normalsize \vspace{2mm} \\
\textbf{Hyun-Jung  Kim} \\ 
\textit{Korea Advanced Institute of Science and Technology} \vspace{{2mm}} \\
\textbf{Mengchu  Zhou} \\ 
\textit{New Jersey Institute of Technology}

Navigating the escalating demand for customized products while upholding stringent quality standards presents formidable challenges in the effective management of manufacturing systems. These challenges can be effectively met through the realization of autonomous manufacturing, minimizing human intervention. Autonomous manufacturing represents a transformative shift towards intelligent, self-governing production systems that leverage cutting-edge technologies to enhance efficiency, flexibility, and responsiveness within manufacturing processes. This necessitates the imperative adoption of advanced technologies such as reinforcement learning, deep learning, meta-learning, digital twin, and optimization methods. The urgency for autonomous manufacturing is particularly pronounced in industries like semiconductors, displays, batteries, and biopharmaceuticals, where production requirements encompass a myriad of complexities, ranging from time window constraints and limited buffers to sequence-dependent setup times, transport robots, and stringent quality control measures. To address these intricacies, this special session aims to bring together researchers, engineers, scientists, and managers actively involved in the research, development, and operation of manufacturing systems. The objective is to foster a collaborative effort to tackle the diverse challenges associated with modeling, scheduling, operation, and control for the advancement of autonomous manufacturing. 

\section{Advances in Intelligent Healthcare Management}


\large \textbf{Organizers} \normalsize \vspace{2mm} \\
\textbf{Zhibin  Jiang} \\ 
\textit{Shanghai Jiao Tong University} \vspace{{2mm}} \\
\textbf{Xiaolan  Xie} \\ 
\textit{Ecole Des Mines De Saint Etienne} \vspace{{2mm}} \\
\textbf{Liping  Zhou} \\ 
\textit{Shanghai Jiao Tong University}

The healthcare industry in many countries is facing significant challenges of rising costs, limited resources, increasing patient demand, and complex healthcare systems. Intelligent healthcare system integrating emerging technologies such as electronic health records, IoT devices, robotics and automation, AI, and telehealth and online healthcare might provide a promising solution. These technologies also raise challenges to healthcare management in terms of workflow adjustment, streamlining processes, data management, interoperability, ethics, privacy and security, medical quality, and cost and resource management that need to be addressed for effective implementation and utilization. In order for emerging technologies to better empower healthcare delivery systems, new methods and tools are being innovated to transform healthcare system by improving efficiency, decision-making, and patient care outcomes.  This special session aims to explore the advancements in intelligent healthcare management by researchers, practitioners, and policymakers from different disciplines. In this session, we welcome all contributions that relate to using the modeling and algorithmic approaches and software tools in operations management, medical decision, AI, data analytics, and policy making with application to addressing intelligent healthcare management problems. 

\section{AI Enabled Discrete Event Dynamic Systems}


\large \textbf{Organizers} \normalsize \vspace{2mm} \\
\textbf{Qianchuan  Zhao} \\ 
\textit{Tsinghua University} \vspace{{2mm}} \\
\textbf{Kai  Cai} \\ 
\textit{Osaka Metropolitan University} \vspace{{2mm}} \\
\textbf{Xiang  Yin} \\ 
\textit{Shanghai Jiao Tong Univ} \vspace{{2mm}} \\
\textbf{Li  Xia} \\ 
\textit{Sun Yat-Sen University}

Discrete event dynamic systems (DEDS) aim at studying the man-made systems driven by events, such as the systems of manufacturing, transportation, computer, communication, energy, robots, etc. The foundation of DEDS is built on mathematical models, such as Markov models, Petri net, automata, queueing models, etc. The decision and control of DEDS is fundamental to improve the operation efficiency of those man-made systems, which involves the optimization theory such as Markov decision process (MDP), optimal control, supervisory control, etc. Recently, the remarkable successes of AI attract intensive attention on the study of data-driven learning and optimization. One of the main research streams of AI is to handle the dynamic decision-making problem with reinforcement learning, whose mathematical foundation is MDP. Therefore, with these facts, the research development of DEDS theory encounters a crossroad, combining the techniques of AI and enabling the study of DEDS in a manner of data-driven learning and optimization.  This special session aims to bring together the international scholars and industry practitioners to discuss the recent progress of DEDS in the background of big development of AI techniques, while focusing on the field of automation science and engineering. The potential topics include but are not limited to the development of DEDS theory such as Markov systems, Petri net, automata, the development of reinforcement learning \& MDP decision theory, the AI enabled solution to dynamic games \& multi-agent systems, and the application of above theories to solve engineering problems in the field of automation science and engineering. 

\section{Assembly Lines in Circulation: A Focus on Human-Centric, Sustainable, and Resilient Manufacturing}


\large \textbf{Organizers} \normalsize \vspace{2mm} \\
\textbf{Hamidreza  Rezaei} \\ 
\textit{IMT Atlantique} \vspace{{2mm}} \\
\textbf{Simon  Thevenin} \\ 
\textit{IMT Atlantique} \vspace{{2mm}} \\
\textbf{Hashemi Petroodi  S.Ehsan} \\ 
\textit{KEDGE Business School}

In the current era of intense global competition and rapid technological advances, manufacturers must frequently introduce new product variants. Each change in the product family requires the adaptation of manufacturing systems, and the frequent reconfiguration of production lines leads to sustainability challenges. Production resources are prematurely taken out of service, leading to scrapping or, at best, being sold for spare parts.  In response to these challenges, the concepts of Industry 4.0 and Industry 5.0 have been introduced. Industry 4.0 represents significant improvements in production efficiency and flexibility through digitization and AI-driven technologies. Industry 5.0 takes these technological advances further, and it focuses on three fundamental principles: sustainability, human-centricity, and resilience. By emphasizing these pillars, Industry 5.0 seeks to address the environmental impact of manufacturing processes, improve the well-being of workers, and fortify systems against unforeseen disruptions. Despite its promising potential, the exploration of Industry 5.0 is still in its early stages, with limited research and a lack of systematic understanding. In this context, this session welcomes original theoretical approaches and new applications that improve sustainability, resilience, and human centricity during the design, maintenance, operation, and decommissioning of assembly lines. 

\section{Autonomous Systems for Agriculture and Horticulture}


\large \textbf{Organizers} \normalsize \vspace{2mm} \\
\textbf{David A.  Anisi} \\ 
\textit{Abb} \vspace{{2mm}} \\
\textbf{Shen Hin  Lim} \\ 
\textit{University of Waikato} \vspace{{2mm}} \\
\textbf{Mike  Duke} \\ 
\textit{University of Waikato} \vspace{{2mm}} \\
\textbf{Benjamin John  Mcguinness} \\ 
\textit{University of Waikato} \vspace{{2mm}} \\
\textbf{Weria  Khaksar} \\ 
\textit{Norwegian University of Life Sciences} \vspace{{2mm}} \\
\textbf{Antonio  Candea Leite} \\ 
\textit{Norwegian University of Life Sciences} \vspace{{2mm}} \\
\textbf{Ibrahim  A. Hameed} \\ 
\textit{NTNU I Ålesund}

Technology advancements, along with multiple significant issues such as lack of labour and general labour intensity in agriculture tasks, have sparked a rapid growth in agriculture and horticulture automation and robotics over the past 15 years. There are successful demonstrations in different aspects of agriculture and horticulture tasks such as crop and fruit harvesting, flower pollination and thinning, crop and fruit detection in different cycle phases and smart overall monitoring and management of weeds, crops, and fruits. However, it remains challenging for smart systems to achieve the dexterity, complexity, and speed of human motion in completing agriculture and horticulture tasks with full autonomy capability. In addressing these challenges, this special session aims to reflect on current advancements in automation and robotics technology in agriculture and horticulture that will eventually allow commercial adoption and application in farms, orchards, and greenhouses. The topics include advancements in perception, monitoring and grasping of agriculture and horticulture tasks, fundamental advancements, and applications in enabling full autonomy of ground or aerial vehicles (localization, navigation and mapping) in agriculture and horticulture conditions, systems of systems approach integration, human-robot collaborative work, safety and smart human-assisted tools for agriculture and horticulture tasks. 

\section{Collaborative Robot-Enabled Advanced Manufacturing in the Context of Industry 5.0}


\large \textbf{Organizers} \normalsize \vspace{2mm} \\
\textbf{Weitian  Wang} \\ 
\textit{Montclair State University} \vspace{{2mm}} \\
\textbf{Xiwang  Guo} \\ 
\textit{Liaoning Petrochemical University} \vspace{{2mm}} \\
\textbf{Mengchu  Zhou} \\ 
\textit{New Jersey Institute of Technology}

During the Industry 5.0 revolution, robotics has become increasingly significant in advancing manufacturing automation and intelligence. Different from traditional industrial robots, which are fenced off from human workers on production lines, collaborative robots make a tremendous shift to coexist and collaborate with human workers in open environments for manufacturing tasks. This makes robots capable of democratizing manufacturing industries with dynamic customer demands, high flexibility, and low cost. In the human-robot collaborative advanced manufacturing, humans have unmatched problem-solving skills and develop creative and unique solutions compared to robots, which can provide physical assistance, free human partners from dangerous tasks, and lower the stress of human collaborators. Therefore, the cooperation and interaction between humans and robots for advanced manufacturing should create new potential opportunities and benefits to enhance safety, optimize production, boost efficiency, improve task quality, and increase worker productivity for industry sectors. In both academia and industries, new interdependent and cross-disciplinary research issues arise and need to be addressed to make human-robot teams more productive and ergonomic. This special session aims to bring researchers, engineers, scientists, and managers engaged in frontier research and technologies of robotics, automation, manufacturing, and cyber-physical systems to investigate and solve different open questions in the field of human-robot collaborative advanced manufacturing. Prospective authors are invited to share their state-of-the-art research findings to address the gaps in this area. 

\section{Decision and Control Techniques for Autonomous Systems}


\large \textbf{Organizers} \normalsize \vspace{2mm} \\
\textbf{Laura  Giarrè} \\ 
\textit{Università Di Modena E Reggio Emilia} \vspace{{2mm}} \\
\textbf{Alberto  Cavallo} \\ 
\textit{Seconda Università Degli Studi Di Napoli} \vspace{{2mm}} \\
\textbf{Federica  Pascucci} \\ 
\textit{Università Roma Tre} \vspace{{2mm}} \\
\textbf{Raffaele  Carli} \\ 
\textit{Politecnico Di Bari}

One of the enabling technologies of the digital transition is Autonomous Systems, which are systems capable of automatically achieving a given goal without the intervention of a human operator. They are able to collect information about the environment in which they operate, process data from multiple sources, and determine and plan actions to be taken to optimize performance, ensuring the safety of people and the proper functioning of devices. There is no doubt that they will assume a crucial role in society. The purpose of this Special Session is to propose new and innovative decision and control techniques for developing Autonomous Systems, thus enabling sustainable problem-solving in extremely diverse application fields, such as manufacturing processes, automotive, aerospace, defense, road and rail mobility, smart cities, home automation, energy networks, water networks, environmental monitoring, smart agriculture, logistics, telecommunications, biomedicine, green transition, etc. Original articles from both scholars and practitioners are welcome, reporting on concepts and approaches for Autonomous Systems and addressing system and control engineering topics that include but are not limited to i) design and develop autonomous engineering systems, with applications to smart manufacturing, autonomous vehicles, smart grids, robotics, and many more; ii) develop smart control algorithms (e.g., AI-enabled control, data-driven control, vision-based control) for smart environments such as smart cities, autonomous vehicles and mobile robots, smart grids, sustainable mobility systems, smart buildings, and smart homes, iii) develop testing platforms for emerging techniques to advance engineering autonomous system applications (e.g., cyber-physical systems, digital twin techniques). 

\section{Digital Twin for Smart Engineering System Development, Operation and Optimization}


\large \textbf{Organizers} \normalsize \vspace{2mm} \\
\textbf{Yan  Lu} \\ 
\textit{National Institute of Standards and Technology} \vspace{{2mm}} \\
\textbf{Feng  Ju} \\ 
\textit{Arizona State University} \vspace{{2mm}} \\
\textbf{Haw-Ching  Yang} \\ 
\textit{National Kaohsiung Univ. Of Sci. And Tech.} \vspace{{2mm}} \\
\textbf{Min-Hsiung  Hung} \\ 
\textit{Chinese Culture University}

Digital twin, a concept integrating modeling and data analytics, has become increasingly more relevant to engineering system design, development and operation. Different from traditional model-based engineering, a Digital Twin includes a dynamic, virtual representation of a physical system or object managed through the lifecycle of the physical system to best support various engineering purposes. It continuously mirrors the real-world entity by incorporating real-time data and machine learning, employing multiple-level, multiple-fidelity models and simulations, and applying advanced reasoning and fusion for prediction and decision-making. Digital Twins can be applied in various engineering systems, including space vehicles, manufacturing, medical devices, and smart building and infrastructure. They are particularly useful for predicting performance, process control and optimization and maintaining assets like machines, energy networks or entire cities. This special session explores the frontier challenges and solutions associated with the design and application of digital twins for successfully engineering system development and operation. 

\section{Drivers and Tools for Efficiency Improvement in Industrial and Non-Industrial Processes}


\large \textbf{Organizers} \normalsize \vspace{2mm} \\
\textbf{Silvia Maria  Zanoli} \\ 
\textit{Polytechnical University of Marche} \vspace{{2mm}} \\
\textbf{Crescenzo  Pepe} \\ 
\textit{Università Politecnica Delle Marche}

Continuous efficiency improvement in industrial and non-industrial processes represents a crucial challenge in the current energy and digital transition and geopolitical situation. Efficiency can be associated to different processes, e.g., plants, machineries, and devices; efficiency can be referred to different points of view, e.g., energy and production. Ad hoc powerful drivers and tools are becoming crucial in this context; furthermore, suitable Key Performance Indicators (KPIs) are needed as metrics for evaluation and certification of efficiency. In this field, the impact of automation science and engineering is growing significantly. The fields of application are increasing also thanks to cross-fertilization between different areas: well-established technologies in a defined area can be extended and customized for other areas. Automation science and engineering systems could be located at different levels of the automation hierarchy; tailored drivers and tools can be detected and developed, while profiting from KPIs in order to evaluate and certify performance and benefits through an objective and unbiased paradigm. The present Special Session aims to collect contributions related to drivers and tools for efficiency improvement in industrial and non-industrial processes with a focus on the impact of automation science and engineering in this field. The Special Session would embrace emerging technologies and best practices. The Special Session will take into consideration research works on simulations in virtual environments and field applications. In addition, contributions on methodologies aimed at bridging the gap between simulations and field application are welcome, together with works on the transition from research to large-scale deployment. 

\section{Emerging Data Science in Manufacturing}


\large \textbf{Organizers} \normalsize \vspace{2mm} \\
\textbf{Chia-Yen  Lee} \\ 
\textit{National Taiwan University} \vspace{{2mm}} \\
\textbf{Chia-Yu  Hsu} \\ 
\textit{National Taiwan University of Science and Technology} \vspace{{2mm}} \\
\textbf{Shu-Kai S.  Fan} \\ 
\textit{National Taipei University of Technology} \vspace{{2mm}} \\
\textbf{Feng  Ju} \\ 
\textit{Arizona State University} \vspace{{2mm}} \\
\textbf{Jakey  Blue} \\ 
\textit{National Taiwan University} \vspace{{2mm}} \\
\textbf{Young Jae  Jang} \\ 
\textit{Korea Advanced Institute of Science and Technology} \vspace{{2mm}} \\
\textbf{Anders  Skoogh} \\ 
\textit{Chalmers University of Technology} \vspace{{2mm}} \\
\textbf{Giovanni  Lugaresi} \\ 
\textit{Laboratoire Génie Industriel}

Manufacturing is characterized by capital/labor-intensive, the short product life cycle, rapid technology migration, long production lead-time, and complex production networks. These characteristics bring more challenges and difficulties to the manufacturing management. This session focuses on how the data science or machine learning techniques support problem-solving and enhance the core competence in manufacturing industry. The special session focuses on data science and engineering in the broad area of manufacturing. Theoretical research or empirical study are all welcome. The topics in this session include defect classification, maintenance scheduling, predictive maintenance, and process parameter optimization, etc.  This session would like to provide a platform that offers opportunities to discuss, debate, and exchange ideas, in particular, in a world-side view of manufacturing system. We invite all the researchers, scholars, and graduates when they would like to develop the mathematical/empirical models and benefit the automation and data science field. 

\section{Human-Robot Collaboration for Futuristic Human-Centric Smart Manufacturing}


\large \textbf{Organizers} \normalsize \vspace{2mm} \\
\textbf{Pai  Zheng} \\ 
\textit{The Hong Kong Polytechnic University} \vspace{{2mm}} \\
\textbf{Tao  Peng} \\ 
\textit{Zhejiang University} \vspace{{2mm}} \\
\textbf{Jinsong  Bao} \\ 
\textit{DongHua University} \vspace{{2mm}} \\
\textbf{Shufei  Li} \\ 
\textit{The Hong Kong Polytechnic University} \vspace{{2mm}} \\
\textbf{Wenjun  Xu} \\ 
\textit{Wuhan University of Technology} \vspace{{2mm}} \\
\textbf{Xi Vincent  Wang} \\ 
\textit{KTH Royal Institute of Technology} \vspace{{2mm}} \\
\textbf{George Q.  Huang} \\ 
\textit{The University of Hong Kong} \vspace{{2mm}} \\
\textbf{Lihui  Wang} \\ 
\textit{KTH Royal Institute of Technology} \vspace{{2mm}} \\
\textbf{Duc Truong  Pham} \\ 
\textit{University of Birmingham}

In line with the human-centric concerns of Industry 5.0, modern factories are striving for an ever-higher degree of flexible and resilient production, as conventional automation approach has reached its bottleneck considering mass personalization with increasing complicatedness and complexity. To achieve it, human-robot collaboration (HRC) becomes a prevailing strategy, which combines high accuracy, strength, and repeatability of industrial robots with high flexibility and adaptability of human operators to realise optimal overall productivity. Cutting-edge technologies, including robot learning and control, cognitive computing, mixed reality/metaverse, industrial IoT, and advanced data analytics create the potentials to bridge the gap of knowledge distilling and information sharing between onsite operators, robots and the manufacturing system with mutual cognitions. Therefore, this special session aims to bring together specialists in different fields of manufacturing systems, robotics, artificial intelligence, and other engineering domains to address the foreseeable HRC-empowered human-centric smart manufacturing paradigm characterized with high-level teamwork skills. 

\section{Industrial Foundation Models and Applications in Smart Manufacturing}


\large \textbf{Organizers} \normalsize \vspace{2mm} \\
\textbf{Weiming  Shen} \\ 
\textit{Huazhong University of Science and Technology} \vspace{{2mm}} \\
\textbf{Mengchu  Zhou} \\ 
\textit{New Jersey Institute of Technology} \vspace{{2mm}} \\
\textbf{Giacomo  Boracchi} \\ 
\textit{Politecnico Di Milano} \vspace{{2mm}} \\
\textbf{Kai  Xu} \\ 
\textit{National University of Defense Technology} \vspace{{2mm}} \\
\textbf{Yunkang  Cao} \\ 
\textit{Huazhong University of Science and Technology}

Foundation models, characterized by extensive parameters and large-scale training data, stand as strong tools equipped with broad knowledge gleaned from training data. These foundation models showcase prowess in reasoning, abstraction, and interaction in common fields. Despite their versatile capabilities, existing foundation models face challenges in specific domains such as smart manufacturing, primarily due to the absence of domain-specific knowledge. In the domain of smart manufacturing, contemporary factories aspire to elevate flexibility, quality, and intelligence in their production processes. Conventional automation methods, constrained in interaction, flexibility, and decision-making, necessitate transformative solutions. Here, foundation models emerge as valuable solutions with their substantial advancements. This special session is dedicated to Industrial Foundation Models (IFMs), with the aim of enriching foundation models with more domain-specific knowledge tailored for industrial scenarios. The primary objective is to navigate the intricacies of IFM development, addressing challenges related to processing extensive industrial data and formulating innovative training schemes tailored to the unique demands of industrial scenarios. These training schemes may include training IFMs from scratch, fine-tuning existing foundation models for specific industrial domains, new use of foundation models in the industry, and more. With a focus on smart manufacturing, this session explores the potential of IFMs in automating processes, enhancing capabilities for both humans and machines, with particular applications to product quality monitoring, prognostics and health management, and operations management. 

\section{Innovations in Robotics and Automation for Enhanced Healthcare}


\large \textbf{Organizers} \normalsize \vspace{2mm} \\
\textbf{Dario  Sanalitro} \\ 
\textit{University of Catania} \vspace{{2mm}} \\
\textbf{Enrico  Ferrentino} \\ 
\textit{University of Salerno} \vspace{{2mm}} \\
\textbf{Loris  Roveda} \\ 
\textit{Supsi-Idsia} \vspace{{2mm}} \\
\textbf{Alessandro  Palleschi} \\ 
\textit{Floating Robotics}

Objectives: This special session aims to explore recent advancements and challenges in integrating robotics and automation into healthcare practices. This session seeks to address the latest developments in diagnostic and therapeutic applications, minimally invasive surgery, rehabilitation, daily-life assistance, and efficient management of healthcare procedures. It provides a platform for interdisciplinary discussions among researchers.   Justifications:  The session is justified by the need to foster collaboration and share insights in the rapidly evolving field of robotics and automation applied to healthcare. Robotics and automation can significantly enhance patient care by providing precise and efficient medical procedures, reducing human error and complications, and enabling faster patient recovery times. In cases where the availability of healthcare professionals is short, robotics and automation technologies can optimize workforce utilization, through e.g., telemedicine, faster surgical procedures, and shorter hospital stays. This session intends to provide a platform to discuss advancements, share the latest findings, and address the ethical and practical considerations associated with automation and robotics in healthcare.   Outcomes: Participants will gain insights into research on robots operating in the human body, surgical robotics, rehabilitation robotics, daily-life and personal assistance devices, wearable technologies, and cognitive robotics. The session aims to inspire collaborative research initiatives and to accelerate the advancements of healthcare robotics. 

\section{Intelligent Operation, Maintenance, and Scheduling in Complex Systems}


\large \textbf{Organizers} \normalsize \vspace{2mm} \\
\textbf{Ziyan  Zhao} \\ 
\textit{Northeastern University} \vspace{{2mm}} \\
\textbf{Shixin  Liu} \\ 
\textit{Northeastern University} \vspace{{2mm}} \\
\textbf{Kaixiang  Peng} \\ 
\textit{University of Science and Technology Beijing} \vspace{{2mm}} \\
\textbf{Mengchu  Zhou} \\ 
\textit{New Jersey Institute of Technology} \vspace{{2mm}} \\
\textbf{Kai  Zhang} \\ 
\textit{University of Science and Technology Beijing} \vspace{{2mm}} \\
\textbf{Shuwei  Zhu} \\ 
\textit{Jiangnan University}

Operational optimization of complex systems is a multifaceted challenge that necessitates the infusion of intelligent approaches, like machine learning, reinforcement learning, evolutionary computation, and heuristic algorithms. Combining artificial intelligence technology with the characteristics of a specific complex system to solve its intelligent operation, maintenance, and scheduling problems is a hot research field of academia and an important demand of practical application. Related research can improve operation efficiency, ensure safety and stability, reduce energy consumption, and save production costs.  While existing research has made notable strides in tackling these complex issues, numerous related problems persistently remain open, necessitating further investigation and exploration. The research gaps of these challenges present exciting opportunities for researchers and practitioners to delve deeper into uncharted territories and contributing to the ongoing discourse on intelligent system optimization. This special session aims to delve into the intricate realm of intelligent operation, maintenance, and scheduling within complex systems and provide a platform to exchange research results, technical trends, and practical experience related to fault diagnosis, process control, operation research, applied mathematics, and management science. Besides, this session is expected to broaden the intelligent optimization community and promote the application of artificial intelligence in practical complex systems. 

\section{Novel Planning and Control Approaches for Semiconductor Manufacturing}


\large \textbf{Organizers} \normalsize \vspace{2mm} \\
\textbf{Lars  Moench} \\ 
\textit{University of Hagen} \vspace{{2mm}} \\
\textbf{Claude  Yugma} \\ 
\textit{Ecole Des Mines De Saint-Etienne}

Semiconductor manufacturing is one of the most complex manufacturing processes. Due to the increased use of new technologies, the global economy is significantly impacted by the semiconductor industry. Competition in the semiconductor sector is fierce. As a result, the semiconductor industry needs to continually reinvent itself and be resourceful at all decision levels. Efficient design, analysis, and operation of semiconductor wafer manufacturing facilities and corresponding supply chains are essential.  - Various factors can disrupt certain chains of the production networks. For example, the global supply chain has been disrupted by the Covid-19 pandemic which spreads around the world, resulting, for instance, in chip shortage. All this has an impact on the production (product quality, supply, etc.). Subsequently, effective decisions must be taken to conduct production operations at different levels of the supply chain links while addressing disruptions. -To ensure reliable results and reduced operational costs, highly automated manufacturing systems are used to carry out operations and make millions of decisions per day. This requires planning and scheduling methods in manufacturing execution systems (MES) and in logistics/supply chain management tools to support their automated operation. As new trends such as sustainable production, cloud computing, and Industry 4.0 emerge, they also need to be addressed. -The development and application of planning and scheduling methods for these high-cost systems and supply-chains are critical elements in improving their operations. The purpose of the proposed session is to highlight cutting-edge research on semiconductor manufacturing planning, scheduling, and supply-chain management. 

\section{Safe and Secure Human-Machine Interaction: A SMCS Contribution}


\large \textbf{Organizers} \normalsize \vspace{2mm} \\
\textbf{Laura  Giarrè} \\ 
\textit{Università Di Modena E Reggio Emilia} \vspace{{2mm}} \\
\textbf{Giancarlo  Fortino} \\ 
\textit{Università Della Calabria} \vspace{{2mm}} \\
\textbf{Graziana  Cavone} \\ 
\textit{University Roma Tre} \vspace{{2mm}} \\
\textbf{Giuseppe  D'Aniello} \\ 
\textit{University of Salerno}

Systems Man and Cybernetics and Robotics and Automation Societies both encourage interdisciplinary approaches in engineering that spread among various fields, such as computer science, control systems, electrical engineering, mathematics, mechanical engineering, system engineering, human systems, human organizational interactions, cybernetics, communication and control across humans and machines.   This Special Session particularly focuses on human-machine interaction, for which is it is of paramount importance to ensure safety of human beings and security of machines. In fact, improper control of machines might cause accidental harms to operators; and also safety breach can lead the machine to unsafe states and cause damage to the surrounding environment, including operators. The main aim of this session is to collect contributions on the most recent cross-disciplinary techniques to develop Safe and Secure Human-Machine Interaction. Thus, enabling the advancement of modelling, control, and information processing in various related application fields, such as manufacturing, logistics, smart mobility, aerospace, smart cities, networks, etc.  In addition, this session is devoted also to the celebration of the Italian SMC Day 2024 by collecting some of the most significant and advanced research and contributions of SMCS and/or RAS Members on the topic. Original articles are welcome, reporting on theories and methods for Safe and Secure Human-Machine Interaction and addressing system and control engineering topics. 

\section{Smart and Sustainable Manufacturing}


\large \textbf{Organizers} \normalsize \vspace{2mm} \\
\textbf{Qing  Chang} \\ 
\textit{University of Virginia} \vspace{{2mm}} \\
\textbf{Congbo  Li} \\ 
\textit{Chongqing University} \vspace{{2mm}} \\
\textbf{Wei  Wu} \\ 
\textit{Chongqing Univeristy} \vspace{{2mm}} \\
\textbf{Feng  Ju} \\ 
\textit{Arizona State University} \vspace{{2mm}} \\
\textbf{Nicla  Frigerio} \\ 
\textit{Politecnico Di Milano}

Smart manufacturing, in the era of Industry 4.0, leverages advanced technologies such as the Internet of Things (IoT), artificial intelligence (AI), data analytics, robotics, and cyber-physical systems. The objective is to create intelligent, interconnected, and automated manufacturing systems, thus leading to enhanced productivity, improved quality control, reduced downtime, and increased flexibility. Simultaneously, sustainable manufacturing has gained considerable attention as a means to address the environmental, social, and economic challenges associated with conventional manufacturing. Sustainable manufacturing focuses on minimizing resource consumption, reducing waste generation, optimizing energy usage, and adopting eco-friendly production processes. There is a growing inclination among customers and markets towards digital, customized, and flexible solutions with a reduced environmental impact. This inclination aligns with the central concept of Industry 5.0, which complements the Industry 4.0 approach by placing research at the forefront of the transition to a sustainable, human-centric, and resilient industry. The integration of smart and sustainable manufacturing is pivotal, holding the potential to revolutionize the environmental and economic impact of the manufacturing industry.  This session will be an excellent opportunity for networking, collaboration, and knowledge exchange among industry experts, academics, and professionals who are passionate about advancing the field of manufacturing towards a more intelligent and sustainable future. 

\vfill\null